\documentclass[a4paper]{jpconf}
\usepackage{graphicx}
\begin{document}
\title{Fast emulation of track reconstruction in CMS simulation}

\author{Matthias Komm, on behalf of the CMS collaboration}

\address{Centre for Cosmology, Particle Physics and Phenomenology,
Universit\'e catholique de Louvain, Louvain-la-Neuve, BELGIUM}

\ead{Matthias.Komm@cern.ch}

\begin{abstract}
Simulated samples of various physics processes are a key ingredient within analyses to unlock the physics behind LHC collision data. Samples with more and more statistics are required to keep up with the increasing amounts of recorded data. During sample generation, significant computing time is spent on the reconstruction of charged particle tracks from energy deposits which additionally scales with the pileup conditions. In CMS, the Fast Simulation package is developed providing a fast alternative to the standard simulation and reconstruction work flow. It employs various techniques to emulate track reconstruction effects in particle collision events amongst others. Several analysis groups in CMS are utilizing the package, in particular those requiring many samples to scan the parameter space of physics models (e.g. SUSY) or for the purpose of estimating systematic uncertainties. The strategies for and recent developments in this emulation are presented which features a novel, flexible implementation of tracking emulation while retaining a sufficient, tuneable accuracy.
\end{abstract}

\section{Introduction}
Tracking of charged particles is one of the crucial ingredients to understand the physics behind LHC collisions. From reconstructed tracks higher analysis level objects like jets and the missing transverse energy are derived. In the CMS experiment, tracks are utilized even further by matching them to information gathered by other subdetectors improving the overall event reconstruction and resolution.

Sophisticated tracking algorithms are required to reconstruct tracks from the large amount of charged particles transversing the CMS detector at each bunch crossing. A typical instantaneous luminosity of $10^{34}\,\mathrm{cm}^{-2}\mathrm{s}^{-1}$ which was reached in 2016 can lead to more than 500 reconstructed tracks originating from 20--30 proton-proton interactions per event.

The complex, multistep algorithms which reconstruct tracks from energy depositions left by transversing charged particles in the CMS tracker are unsurprisingly very computing-intense. This standard tracking procedure is applied to simulated events as well after those have passed the emulation of the electronic response of the detector. However, at this point in the simulation chain it can be beneficial to employ faster algorithms by utilizing truth-information about the simulated events instead. In the CMS simulation framework this idea together with others has lead to a fast alternative for simulating physics events called \textsc{FastSimulation}. The increasing number of additional ``pileup'' interactions leading to more tracks motivates the use of such a fast alternative further.

Nowadays, the \textsc{FastSimulation} package is already employed  within CMS. A typical use case are searches for beyond the Standard Model physics such as SUSY. In such analyses, multiple signal samples are required reflecting various realizations of a new physics model. Other use cases are the evaluations of the impact from sources of systematic uncertainties on a measurement like a variation of the renormalization and factorization scales or the top quark mass.

This note is organized as follows: First, the steps involved in the standard track reconstruction are described briefly. Then, their alternative implementation within the so-called ``FastSimulation'' package of CMS is detailed with an emphasis on new developments in its tracking emulation. After the validation where the obtained tracking performances are compared this note is summarized and an outlook on planned developments is provided.

\section{Standard tracking within CMS}

\section{Emulation of tracking reconstruction}

\section{Validation}

\section{Summary and outlook}

\ack
blub


\section{References}

\begin{thebibliography}{9}
\item CMS Collaboration 2014, {\it JINST} \textbf{9} no.10, P10009.


%\item Strite S and Morkoc H 1992 {\it J. Vac. Sci. Technol.} B {\bf 10} 1237 
%\item Jain S C, Willander M, Narayan J and van Overstraeten R 2000 
%{\it J. Appl. Phys}. {\bf 87} 965 
%\item Nakamura S, Senoh M, Nagahama S, Iwase N, Yamada T, Matsushita T, Kiyoku H 
%and 	Sugimoto Y 1996 {\it Japan. J. Appl. Phys.} {\bf 35} L74 
%\item Akasaki I, Sota S, Sakai H, Tanaka T, Koike M and Amano H 1996 
%{\it Electron. Lett.} {\bf 32} 1105 
%\item O'Leary S K, Foutz B E, Shur M S, Bhapkar U V and Eastman L F 1998 
%{\it J. Appl. Phys.} {\bf 83} 826 
%\item Jenkins D W and Dow J D 1989 {\it Phys. Rev.} B {\bf 39} 3317 
\end{thebibliography}

\end{document}


