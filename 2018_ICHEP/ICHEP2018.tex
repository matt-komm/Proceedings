
\documentclass{PoS}
\usepackage{amsmath}

\title{Single top quark production in CMS}

\ShortTitle{Single top quark production in CMS}

\author{
    \speaker{Matthias Komm}, on behalf of the CMS collaborations\\
    Imperial College London (UK)\\
    E-mail: \email{Matthias.Komm@cern.ch}
}

\abstract{
In this note, latest cross section measurements of single top production in the three main production modes, $s$-channel, $t$-channel, and W-associated by the CMS collaboration are presented using proton-proton collision data at centre-of-mass energies of 8 and 13~TeV.
}

\FullConference{
    39th International Conference on High Energy Physics (ICHEP)\\
    04-11 July, 2018\\
    Seoul, Korea
}

\begin{document}

\section{Introduction}
The single top quark production processes are excellent probes to test the predictions of electroweak interactions at the scale of the top quark mass and beyond. Inclusive single top quark cross section measurements can be used to infer the absolute value of the CKM matrix element $\mathrm{V}_\mathrm{tb}$ in a model-independent manner whereas  ratios of top quark over antiquark production cross sections are sensitive to the parton distribution function (PDF). In the following recent measurements of single top quark production via the $s$~channel, W-associated, and $t$-channel are presented.

\section{\textit{s}~channel}

The single top quark $s$-channel production mode has the smallest cross section amongst the three processes. It has been measured using pp collision data recorded by the CMS experiment at centre-of-mass energies of 7 and 8~TeV~\cite{sch}. Events containing one isolated electron or muon and two or three jets of which one or two are b-tagged have been analysed. By performing a simultaneous maximum likelihood (ML) fit the the distribution of a Boosted Decision Tree (BDT) discriminant a signal strength of $\sigma^\mathrm{meas.}_{s\mbox{-}\mathrm{ch.}}/\sigma^\mathrm{theo.}_{s\mbox{-}\mathrm{ch.}}=2.0\pm0.9$ is obtained. This result corresponds to an observed (expected) significance of 2.5 (1.1) standard deviations.

%\begin{figure}[!htb]
%\begin{center}
%\includegraphics[width=0.48\textwidth]{sch1.pdf}\hspace{0.02\textwidth}
%\includegraphics[width=0.48\textwidth]{sch2.pdf}
%\caption{\label{fig:s-channel-bdt}Ref.~\cite{sch}.}
%\end{center}
%\end{figure}

\section{W-associated production}

The cross section of producing a single top quark in association with a W~boson has been measured at 13~TeV in events containing one isolated muon and one isolated electron together with one or two jets~\cite{tWch}. The signal yield is estimated by performing a ML fit to the distributions of a BDT discriminant in 1j1b and 2j1b regions and to the transverse momentum of the subleading jet in the 2j2b control region where the latter allows an in-situ constraint of the jet energy scale. A cross section of $63.1\pm1.8~\mathrm{(stat)}\pm6.4~\mathrm{(syst)}\pm2.1~\mathrm{(lumi)}~\mathrm{pb}$ is found which agrees well with the predicted cross section of $71.7\pm1.8~\mathrm{(scale)}\pm3.4~\mathrm{(PDF)}~\mathrm{pb}$ calculated at approximate next-to-next-to-leading order~\cite{tw-xsec}.

%\begin{figure}[!htb]
%\begin{center}
%\includegraphics[width=0.48\textwidth]{tw2.pdf}\hspace{0.02\textwidth}
%\includegraphics[width=0.48\textwidth]{tw3.pdf}
%\caption{\label{fig:tw-channel-bdt}Distributions of a BDT discriminant in (left)~1j1b and (right)~2j1b region used to estimate the W-associated single top quark cross section at 13~TeV. The figures are taken from Ref.~\cite{tWch}.}
%\end{center}
%\end{figure}

\section{\textit{t}~channel}

%\begin{figure}[!htb]
%\begin{center}
%\includegraphics[width=0.48\textwidth]{tch1.pdf}\hspace{0.02\textwidth}
%\includegraphics[width=0.48\textwidth]{tch2.pdf}
%\caption{\label{fig:t-channel-bdt}Ref.~\cite{tch}.}
%\end{center}
%\end{figure}

\begin{figure}[!htb]
\begin{center}
\includegraphics[width=0.6\textwidth]{tch3.pdf}
\caption{\label{fig:t-channel-ratio}Ref.~\cite{tch}.}
\end{center}
\end{figure}

\begin{figure}[!htb]
\begin{center}
\includegraphics[width=0.48\textwidth]{unfolded_top_pt.pdf}\hspace{0.02\textwidth}
\includegraphics[width=0.48\textwidth]{unfolded_top_y.pdf}
\caption{\label{fig:t-channel-diff}Ref.~\cite{tchdiff}.}
\end{center}
\end{figure}

\section{Conclusion}

\clearpage

\begin{thebibliography}{99}

\bibitem{sch}{CMS Collaboration, \emph{Search for s~channel single top quark production in pp collisions at $ \sqrt{s}=$7 and 8~TeV}, JHEP 09 (2016) 027, \texttt{arXiv:1603.02555}.}

\bibitem{tWch}{CMS Collaboration, \emph{Measurement of the production cross section for single top quarks in association with W~bosons in proton-proton collisions at $\sqrt{s}=$13~TeV}, JHEP 10 (2018) 117, \texttt{arXiv:1805.07399}.}

\bibitem{tw-xsec}{N. Kidonakis, \emph{Theoretical results for electroweak-boson and single-top production}, Proceedings PoS DIS 2015, \texttt{arXiv:1506.04072}.}

\bibitem{tch}{CMS Collaboration, \emph{Measurement of the single top quark and antiquark production cross sections in the t~channel and their ratio in pp collisions at $\sqrt{s}=$13~TeV}, CMS-PAS-TOP-17-011, 2018, \texttt{cds.cern.ch/record/2628541}.}

\bibitem{tchdiff}{CMS Collaboration, \emph{Measurement of the differential cross section for t-channel single-top-quark production at $\sqrt{s}=$13~TeV}, CMS-PAS-TOP-16-004, 2016, \texttt{cds.cern.ch/record/2151074}.}

\end{thebibliography}



\end{document}
